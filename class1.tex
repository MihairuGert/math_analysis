\documentclass[a4paper]{article}
\usepackage[T2A]{fontenc}
\usepackage[english, russian]{babel}
\usepackage[utf8]{inputenc}
\usepackage{amsmath}
\usepackage{amssymb}
\usepackage{geometry}
\usepackage{microtype}
\usepackage{ragged2e}
\usepackage{graphicx} % Required for inserting images

\newgeometry{vmargin={15mm}, hmargin={25mm,25mm}}
\pagestyle{empty}

\begin{document}
 \centering\section*{Математический анализ, доп. 1}
 \raggedright
 \section{Вступление}
\justifying
 Перед вами представлен краткий конспект дополнительного занятия. Для более полной информации рекомендуется посещать допы очно.
 \\\\
 Сегодня мы рассмотрим: множества чисел, ограниченность, лемму о плотности, отображения, сюръекцию, инъекцию, биекцию, виды бесконечности(счетная, континуальная), математическую индукцию а также порешаем некоторые примеры. 
\section{Числовые множества}
$\mathbb{N}$ - множество натуральных чисел: $\{1, 2, 3, 4, 5, 6, 7, 8, 9, ...\} $ (часто математики добавляют 0) \\
$\mathbb{Z}$ - множество целых чисел: $\{..., -4, -3, -2, -1, 0, 1, 2, 3, 4, ...\}$ \\
$\mathbb{Q}$ - множество рациональных чисел: $\{\frac{a}{b} \ | \ a \in \mathbb{Z},\ b \in \mathbb{N}\}$\\
$\mathbb{R}$ - множество вещественных(действительных) чисел: как рациональные, так и иррациональные числа\\
$\mathbb{R \setminus Q}$ - множество иррациональных чисел: например $\pi$, $e$ (трансцендентные) или $\sqrt{2}, \sqrt{3}$ (алгебраические)\\\\
\textbf{Аксиома Архимеда:}\\
\textit{Для каждого числа a существует целое число k такое, что a < k.}\\\\
\textbf{Лемма о плотности:}\\\textit{Каковы бы ни были действительные числа a и b (a < b),
существуют рациональное число r и иррациональное число $\gamma$ такие, что a < r < b,
a < $\gamma$ < b.}\\\\
\textbf{Понятие ограниченности:}\\
Пусть $A  \subset \mathbb{R}$. Множество $A$ называется ограниченным cвepху (снизу), если существует действительное число $K (k)$ такое, что $\forall x \in A \ x \leqslant K (x \geqslant k)$. При
этом число $K (k)$ называется верхней (нижней) гранью множества $A$.\\ Наименьшая
среди верхних граней множества $A$ называется точной верхней гранью или \textbf{супремумом} множества $A$ и обозначается через $\sup A$.\\Наибольшая среди нижних граней множества $A$ называется его точной нuжней гранью или \textbf{инфимумом} и обозначается через $\inf A$. \\\\Важно то, что \textit{супремум, что инфинум могут как принадлежать множеству, которое ограничивают, так и не принадлежать.}\\
$\sup$ и $\inf$ имеют полезное свойство:\\
$s \in \mathbb{R} \ s = \sup X \Leftrightarrow s$ - верхняя грань X и $\forall\epsilon > 0 \  \exists x \in X \  | \  x + \epsilon > s$\\
Аналогично для инфинума. Запись этого определения оставляем читателю как упражнение.
\\\\
\textbf{Теорема о существовании верхней и нижней граней:}\\
Всякое непустое ограниченное сверху (снизу) множество действительных чисел имеет точную верхнюю
(нижнюю) грань.
\\\\
\textbf{Лемма о вложенных отрезках:}\\
Пусть дана последовательность числовых отрезков ${[a_n, b_n]},
a_n < b_n$. Эту последовательность будем называть \textbf{системой вложенных отрезков}, если
$\forall n$ выполняются неравенства $a_n \leqslant a_{n+1}$ и $b_{n+1} \leqslant b_n$.\\
Для такой системы выполняются включения
$[a_1, b_1] \supset [a_2, b_2] \supset [a_3, b_3] \supset ... \supset [a_n, b_n] \supset ... $
то есть каждый следующий отрезок содержится в предыдущем. \\\\
Пусть задана система отрезков ${[a_n, b_n]}, n = 1, 2, ...$ Мы скажем, что длина отрезков $[a_n, b_n]$ стремится к нулю с возрастанием n, если для любого
числа $\epsilon > 0$ существует номер $n_\epsilon$ такой, что для всех номеров $n \geqslant n_\epsilon$ выполняется
неравенство $b_n - a_n < \epsilon$.\\\\Данное опрделение возможно пугает, однако ничего сложного в нем нет. Мы берем произвольное число, называем эпсилон, подбираем такой номер $n_\epsilon$, что для любого номера больше, длина отрезка $[a_n,b_n]$ была меньше чем этот эпсилон. Где же тут стремление к нулю? Эпсилон может быть сколь угодно близким числом к 0, таким образом уменьшая эпсилон, всё ближе и ближе приблежаясь к нулю, длина отрезков также уменьшается и вместе с эпсилоном приблежается к нулю. Похожее определение, правда куда более фундаментальное, ещё появится в курсе.\\\\
Для всякой системы вложенных отрезков, по длине стремящихся к
нулю, \textit{существует единственная точка, которая принадлежит всем отрезкам данной
системы.} В будущем мы будем ещё применять данное свойство. 
\section{Отображения}
Пусть $X$ и $Y$ – два произвольных множества. Правило $f$, по
которому для каждого элемента $x \in X$ можно найти \textbf{единственный элемент} $y \in Y$, называется отображением. \\
$D(f) = \{x \in X \ | \  \exists y \in Y : y = f(x)\}$ - область определения\\
$E(f) = \{y \in Y \ | \  \exists x \in X : y = f(x)\}$ - область значений\\\\
Инъективное отображение(отображение в), или \textbf{инъекция}, если $x_1 \neq x_2 \Rightarrow f(x_1) \neq f(x_2)$. Иначе говоря, $f(x_1) = f(x_2) \Rightarrow x_1 = x_2$.
\\
Cюръективное отображение(отображение на), или \textbf{сюръекция}, $f:X \xrightarrow 
\  Y$ если $\forall y \in Y \ \exists x \in X : f(x) = y$.\\
Биективное отображение(взаимно-однозначное), или \textbf{биекция}, если инъекция + сюръекция.\\\\
Множество называется \textbf{счетным}, если можно установить биекцию с множеством натуральных чисел.\\
Множество называется \textbf{континуальным}, если можно установить биекцию с множеством вещественных чисел.\\
Вообще говоря, количество бесконечных мощностей - бесконечно(более подробно этот вопрос будет изучаться в курсе математической логики).
\section{Метод математической индукции}
Данный метод будет крайне часто применяться на протижении всего 1 курса, поскольку он крайне удобен в доказательстве различных выражений.\\
Пусть S - некоторое непустое множество целых чисел. Если S ограничено снизу(сверху), то S содержит наименьший(наибольший) элемент. То есть инфимум(супремум) этого множества принадлежит самому множеству. Это утверждение лежит в основе метода математической индукции.\\\\
В первую очередь мы выбираем \textbf{базу}, какое-то число, с которого мы по сути запустим индукцию, и доказываем выражение для неё. Далее мы делаем \textbf{шаг индукции}, переходя от произвольного номера, условно говоря, i-того к i+1-ому, полагая, что при i-том номере выражение было выполнено. Если выражение было выполнено и при базе, и при шаге, то оно верно. Рассмотрим на примере:\\\\
$1 + 2 + 3 + ... + n = \frac{n(n+1)}{2}$(Демидович, 1)\\
База: $n = 1: 1 = \frac{1(1+1)}{2} = 1$ - Выполнена.\\
Шаг: $k \xrightarrow \  k + 1: 1 + 2 + 3 + ... + k + (k + 1)$. По индукционному предположению для номеров меньших выражение выполняется, поэтому сгруппируем первые k членов. $\frac{k(k+1)}{2} + k + 1 = \frac{k(k+1)}{2} + \frac{2k+2}{2} = \frac{k(k+1) + 2(k+1)}{2} = \frac{(k+1)(k+2)}{2}$. Выполнен. Стало быть формула верна.
\\\\
Как такового доказательства этот метод не имеет, однако увидеть, почему он работает - можно, проведем аналогию с домино. База это первая доминошка, если мы её доказали, то эта доминошка упала. Далее в шаге мы доказываем, что если упала какая-то произовольная доминошка, то и следующая за ней тоже падает. Таким обраом, когда падает первая доминошка, падает и вторая, за ней третья и т.д. Доказав базу (например, при n = 1), доказывается и при n = 2, n = 3 и т.д. В итоге получаем доказательтво для всех n. 
\section{Упражнения}
\textbf{Отображения}:\\
\textbf{1) Доказать:}\\
а) $f(A_1 \cap A_2) \subset f(A_1) \cap f(A_2)$\\
Доказательство: Доказываем в две стороны,\\
$\subset$)$ y \in f(A_1 \cap A_2) \Rightarrow \exists x \in (A_1 \cap A_2) : f(x) = y \Leftrightarrow \exists x \in A_1 : f(x) = y$ и $\exists x \in A_2 : f(x) = y \Rightarrow y \in f(A_1)$ и $y \in f(A_2) \Rightarrow y \in f(A) \cap f(B)$\\
$\nsupseteq)$ Подбирем контр-пример, возьмем отображение $f: x \xrightarrow \ x^2 \ A_1 = [0,2] \ A_2 = [-1, 0] \Rightarrow A_1 \cap A_2 = \{0\} \Rightarrow f(A_1 \cap A_2) = 0$. С другой стороны $f(A_1) = [0,4] \ f(A_2) = [0,1] \Rightarrow f(A_1) \cap f(A_2) = [0,1]$. Очевидно, $\{0\} \nsupseteq [0,1]$\\
б) $f(A_1 \cup A_2) = f(A_1) \cup f(A_2)$\\
\textbf{2) Доказать:}\\
a) Множество четных натуральных чисел счетно.
Доказательство: Просто придумаем взаимнооднозначное отображение:\\
$f: \mathbb{N} \xrightarrow \ \mathbb{N}_2: x \xrightarrow \  2x$\\
б) Множество целых чисел счетно.\\
в) Множество рациональных чисел счетно.\\
\textbf{3) Доказать:}\\
а)$(0,1) \longleftrightarrow \ (a,b)$\\
б)$(0,1)$ - континуально\\
Подсказка: $\tg(\pi x - \frac{\pi}{2})$\\
в)$[0,1] \longleftrightarrow \ [0,1]$\\
Доказательство:\\
$f(x):\\
\begin{cases} \frac{1}{2} \xrightarrow \ 1 \\
\frac{1}{4} \xrightarrow \ 0\\ 
\frac{1}{n} \xrightarrow \ \frac{1}{n+1}\\
x \xrightarrow \ x\\
\end{cases}$\\\\
\textbf{Матиндукция}\\
1, Д. 2) $1^2 + 2^2 + 3^2 + ... + n^2 = \frac{n(n+1)(2n+1)}{6}$\\
2, Д. 7) (Неравенство Бернулли) $x > -1,\ n > 1 :\ (1+x)^n \geqslant 1+nx$\\
Доказательство:
База $n=2$)$(1+x)^2 \geqslant 1+2x \Leftrightarrow 1+2x+x^2 \geqslant 1+2x$ Выполнено.\\
Шаг $k \xrightarrow \ k+1$) $(1+x)^{k+1} = (1+x)^k*(1+x) \geqslant (1+kx)(1+x)$ Далее упражнение.\\
3, Д. 10) $\frac{1}{2}*\frac{3}{4}*...*\frac{2n-1}{2n} < \frac{1}{\sqrt{2n+1}}$\\\\
Занятие подготовил: Пятанов М.Ю.
\end{document}
