\documentclass[a4paper]{article}
\usepackage[T2A]{fontenc}
\usepackage[english, russian]{babel}
\usepackage[utf8]{inputenc}
\usepackage{amsmath}
\usepackage{amssymb}
\usepackage{geometry}
\usepackage{microtype}
\usepackage{ragged2e}
\usepackage{graphicx} % Required for inserting images

\newgeometry{vmargin={15mm}, hmargin={25mm,25mm}}
\pagestyle{empty}

\begin{document}
 \centering\section*{Математический анализ, доп. 1}
 \raggedright
 \section{Вступление}
\justifying
 Перед вами представлен краткий конспект дополнительного занятия. Для более полной информации рекомендуется посещать допы очно. Данные записи служат больше для повторения основных моментов.
 \\\\
 Сегодня мы рассмотрим: множества чисел, ограниченность, лемму о плотности, отображения, сюръекцию, инъекцию, биекцию, виды бесконечности(счетная, континуальная), а также порешаем некоторые примеры. 
\section{Множества}
$\mathbb{N}$ - множество натуральных чисел: 1 2 3 4 5 6 7 8 9 ... (Прим. в матлоге с 0) \\
$\mathbb{Z}$ - множество целых чисел: ... -4 -3 -2 -1 0 1 2 3 4 ... \\
$\mathbb{Q}$ - множество рациональных чисел: $\mathbb{N \cup Z} \cup \{$конечные или бесконечные периодические дроби$\}$ \\
$\mathbb{R}$ - множество вещественных(действительных) чисел: как рациональные, так и иррациональные числа\\
$\mathbb{R \setminus Q}$ - множество иррациональных чисел: бесконечные непериодические дроби, например $\pi$, $e$ или $\sqrt{2}$\\
\end{document}
