\documentclass[a4paper]{article}
\usepackage[T2A]{fontenc}
\usepackage[english, russian]{babel}
\usepackage[utf8]{inputenc}
\usepackage{amsmath}
\usepackage{amssymb}
\usepackage{geometry}
\usepackage{microtype}
\usepackage{ragged2e}
\usepackage{graphicx} % Required for inserting images

\newgeometry{vmargin={15mm}, hmargin={25mm,25mm}}
\pagestyle{empty}

\begin{document}
 \centering\section*{Математический анализ, доп. 1}
 \raggedright
 \section{Вступление}
\justifying
 Перед вами представлен краткий конспект дополнительного занятия. Для более полной информации рекомендуется посещать допы очно.
 \\\\
 Сегодня мы рассмотрим: множества чисел, ограниченность, лемму о плотности, отображения, сюръекцию, инъекцию, биекцию, виды бесконечности(счетная, континуальная), математическую индукцию а также порешаем некоторые примеры. 
\section{Числовые множества}
$\mathbb{N}$ - множество натуральных чисел: $\{1, 2, 3, 4, 5, 6, 7, 8, 9, ...\} $ (часто математики добавляют 0) \\
$\mathbb{Z}$ - множество целых чисел: $\{..., -4, -3, -2, -1, 0, 1, 2, 3, 4, ...\}$ \\
$\mathbb{Q}$ - множество рациональных чисел: $\{\frac{a}{b} \ | \ a \in \mathbb{Z},\ b \in \mathbb{N}\}$\\
$\mathbb{R}$ - множество вещественных(действительных) чисел: как рациональные, так и иррациональные числа\\
$\mathbb{R \setminus Q}$ - множество иррациональных чисел: например $\pi$, $e$ (трансцендентные) или $\sqrt{2}, \sqrt{3}$ (алгебраические)\\\\
\textbf{Аксиома Архимеда:}\\
\textit{Для каждого числа a существует целое число k такое, что a < k.}\\\\
\textbf{Лемма о плотности:}\\\textit{Каковы бы ни были действительные числа a и b (a < b),
существуют рациональное число r и иррациональное число $\gamma$ такие, что a < r < b,
a < $\gamma$ < b.}\\\\
\textbf{Понятие ограниченности:}\\
Пусть $A  \subset \mathbb{R}$. Множество $A$ называется ограниченным cвepху (снизу), если существует действительное число $K (k)$ такое, что $\forall x \in A \ x \leqslant K (x \geqslant k)$. При
этом число $K (k)$ называется верхней (нижней) гранью множества $A$.\\ Наименьшая
среди верхних граней множества $A$ называется точной верхней гранью или \textbf{супремумом} множества $A$ и обозначается через $\sup A$.\\Наибольшая среди нижних граней множества $A$ называется его точной нuжней гранью или \textbf{инфимумом} и обозначается через $\inf A$. \\\\Важно то, что \textit{супремум, что инфинум могут как принадлежать множеству, которое ограничивают, так и не принадлежать.}\\
$\sup$ и $\inf$ имеют полезное свойство:\\
$s \in \mathbb{R} \ s = \sup X \Leftrightarrow s$ - верхняя грань X и $\forall\epsilon > 0 \  \exists x \in X \  | \  x + \epsilon > s$\\
Аналогично для инфинума. Запись этого определения оставляем читателю как упражнение.
\\\\
\textbf{Теорема о существовании верхней и нижней граней:}\\
Всякое непустое ограниченное сверху (снизу) множество действительных чисел имеет точную верхнюю
(нижнюю) грань.
\\\\
\textbf{Лемма о вложенных отрезках:}\\
Пусть дана последовательность числовых отрезков ${[a_n, b_n]},
a_n < b_n$. Эту последовательность будем называть \textbf{системой вложенных отрезков}, если
$\forall n$ выполняются неравенства $a_n \leqslant a_{n+1}$ и $b_{n+1} \leqslant b_n$.\\
Для такой системы выполняются включения
$[a_1, b_1] \supset [a_2, b_2] \supset [a_3, b_3] \supset ... \supset [a_n, b_n] \supset ... $
то есть каждый следующий отрезок содержится в предыдущем. \\\\
Пусть задана система отрезков ${[a_n, b_n]}, n = 1, 2, ...$ Мы скажем, что длина отрезков $[a_n, b_n]$ стремится к нулю с возрастанием n, если для любого
числа $\epsilon > 0$ существует номер $n_\epsilon$ такой, что для всех номеров $n \geqslant n_\epsilon$ выполняется
неравенство $b_n - a_n < \epsilon$.\\\\Данное опрделение возможно пугает, однако ничего сложного в нем нет. Мы берем произвольное число, называем эпсилон, подбираем такой номер $n_\epsilon$, что для любого номера больше, длина отрезка $[a_n,b_n]$ была меньше чем этот эпсилон. Где же тут стремление к нулю? Эпсилон может быть сколь угодно близким числом к 0, таким образом уменьшая эпсилон, всё ближе и ближе приблежаясь к нулю, длина отрезков также уменьшается и вместе с эпсилоном приблежается к нулю. Похожее определение, правда куда более фундаментальное, ещё появится в курсе.\\\\
Для всякой системы вложенных отрезков, по длине стремящихся к
нулю, \textit{существует единственная точка, которая принадлежит всем отрезкам данной
системы.} В будущем мы будем ещё применять данное свойство. 
\section{Отображения}
\end{document}
